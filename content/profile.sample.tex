% ========================================
% プロフィール情報設定ファイル(サンプル)
% ========================================
% このファイルをコピーして profile.tex として使用してください

% 基本情報
\newcommand{\profileName}{Your Name}
\newcommand{\profileStudentID}{Your Student ID}
\newcommand{\profileLaboratory}{Your Laboratory Name}
\newcommand{\profileProgram}{Your Program Name}
\newcommand{\profileSchool}{Your School Name}
\newcommand{\profileUniversity}{Your University Name}

% 連絡先情報(オプション)
\newcommand{\profileEmail}{your.email@example.com}
\newcommand{\profilePhone}{+81-xx-xxxx-xxxx}

% 研究分野・専門(オプション)
\newcommand{\profileResearchArea}{Your Research Area}
\newcommand{\profileKeywords}{Keyword1, Keyword2, Keyword3}

% 報告書タイトル
\newcommand{\reportTitle}{Your Report Title}

% 報告書の種類(オプション)
\newcommand{\reportType}{研究報告書}
\newcommand{\reportDate}{\today}

% ========================================
% プロフィール情報を表示するコマンド
% ========================================

% 著者情報を生成
\newcommand{\generateAuthor}{
  \profileName \\
  Student ID: \profileStudentID \\
  \profileLaboratory \\
  \profileProgram \\
  \profileSchool \\
  \profileUniversity
}

% 完全なプロフィール情報を表示(オプション)
\newcommand{\showFullProfile}{
  \begin{center}
    \textbf{\large 著者情報}
  \end{center}
  
  \begin{table}[h]
    \centering
    \begin{tabular}{ll}
      \textbf{氏名:} & \profileName \\
      \textbf{学籍番号:} & \profileStudentID \\
      \textbf{研究室:} & \profileLaboratory \\
      \textbf{プログラム:} & \profileProgram \\
      \textbf{研究科:} & \profileSchool \\
      \textbf{大学:} & \profileUniversity \\
      \textbf{研究分野:} & \profileResearchArea \\
      \textbf{キーワード:} & \profileKeywords \\
      \textbf{連絡先:} & \profileEmail \\
    \end{tabular}
  \end{table}
} 