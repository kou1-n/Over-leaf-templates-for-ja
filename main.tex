\documentclass[dvipdfmx]{article}
\usepackage{graphicx}
% 画像ファイルの場所をLaTeXに教える (今後 \includegraphics で figures/ と書く必要がなくなります)
\graphicspath{{figures/}} % <-- 追加

\usepackage{amsmath}
\usepackage{amssymb}
\usepackage{subcaption}
\usepackage{listings}
\usepackage{xcolor}
\lstset{
  basicstyle=\small\ttfamily,
  commentstyle=\color{green},
  keywordstyle=\color{blue},
  stringstyle=\color{red},
  frame=tb,
  breaklines=true,
  columns=fullflexible,
  numbers=left,
  numberstyle=\tiny\color{gray},
  showstringspaces=false
}
\usepackage{algorithm}
\usepackage{algpseudocode}
% --- フロート制御パッケージ---
% [section]オプションで、\section の前に自動でバリアを設置
\usepackage[section]{placeins}
% \subsection の前に自動でバリアを設置するためのパッケージ
\usepackage{etoolbox}
\preto{\subsection}{\FloatBarrier}
% ------------------------------------

% ----- ハイパーリンクの設定 -----
\usepackage[
  dvipdfmx,
  colorlinks=true,
  linkcolor=blue,
  citecolor=green,
  urlcolor=magenta,
  bookmarks=true,
  bookmarksnumbered=true
]{hyperref}
% -----------------------------

\title{Your Report Title}
\author{Koichi Nagasaku \\
        Student ID: 25HJ046 \\
        Yamada-Matsui Laboratory \\
        Information Science Program \\
        Graduate School of Environment and Information Sciences \\
        Yokohama National University}
\date{\today}

\begin{document}

\maketitle

% ----- 要旨(Abstract) -----
\begin{abstract}
  ここにレポートの要旨を記述します。研究の背景、目的、手法、結果、結論などを200〜400字程度で簡潔にまとめます。
この部分は独立したページに表示されることが多いです。
 % <-- 変更
\end{abstract}
% --------------------------

% ----- 目次、図目次、表目次 -----
\tableofcontents
\listoffigures
\listoftables
\newpage
% -----------------------------

% ----- 本文 -----


\section{はじめに}

本文の執筆は、このファイルで行います。
LaTeXは高機能な文書作成システムであり\cite{lamport1994}、特に数式や参考文献の扱いに優れています。
詳細は公式サイト \href{https://www.latex-project.org/}{The LaTeX Project} や、簡単な紹介サイト \url{https://texwiki.texjp.org/} などで確認できます。




\section{基本的な要素の例}

\subsection{図の挿入}
`figure`環境を用いて、図\ref{fig:sample}のような画像を挿入できます。
`main.tex`に`\graphicspath{{figures/}}`と記述したため、画像ファイルは`figures`フォルダに置いてください。

\begin{figure}[htbp]
  \centering
  \includegraphics[width=0.7\textwidth]{example-image-a}
  \caption{基本的な図の例}
  \label{fig:sample}
\end{figure}

\subsection{表の作成}
表\ref{tab:sample}に、`tabular`環境を用いて作成した表の例を示します。

\begin{table}[htbp]
  \centering
  \caption{サンプル表}
  \label{tab:sample}
  \begin{tabular}{|l|c|r|}
    \hline
    \textbf{左揃え (l)} & \textbf{中央揃え (c)} & \textbf{右揃え (r)} \\
    \hline
    商品A & 150個 & 30,000円 \\
    商品B & 20個 & 5,400円 \\
    商品C & 350個 & 105,000円 \\
    \hline
  \end{tabular}
\end{table}

\subsection{グラフの挿入}
グラフも図と同様に、外部ツールで作成した画像ファイルを挿入するのが一般的です(図\ref{fig:graph-image})。

\begin{figure}[htbp]
  \centering
  \includegraphics[width=0.8\textwidth]{example-image-b}
  \caption{外部ツールで作成し、画像として挿入したグラフの例}
  \label{fig:graph-image}
\end{figure}




\section{技術的な要素の記述例}

\subsection{Subfigureによる図の並列配置}
`subcaption`パッケージを使うと、図\ref{fig:subfigures}のように複数の図を並べて配置し、それぞれに(a), (b)のような副キャプションを付けることができます。

\begin{figure}[htbp]
  \centering
  % 1つ目の図 (a)
  \begin{subfigure}[b]{0.48\textwidth}
    \centering
    \includegraphics[width=\textwidth]{example-image-a}
    \caption{1つ目の図(結果A)}
    \label{fig:sub-a}
  \end{subfigure}
  \hfill % 図の間に少しスペースを空ける
  % 2つ目の図 (b)
  \begin{subfigure}[b]{0.48\textwidth}
    \centering
    \includegraphics[width=\textwidth]{example-image-b}
    \caption{2つ目の図(結果B)}
    \label{fig:sub-b}
  \end{subfigure}
  \caption{2つの結果を比較した図}
  \label{fig:subfigures}
\end{figure}

\subsection{ソースコードの掲載}
`listings`パッケージを用いると、ソースコード\ref{lst:pythonsample}のように、シンタックスハイライトや行番号を付けてコードを綺麗に表示できます。

\begin{lstlisting}[language=Python, caption={Pythonによる簡単なサンプルコード}, label={lst:pythonsample}]
# This is a sample Python code.
import numpy as np

def calculate_mean(data):
    """Calculate the mean of a list of numbers."""
    if not data:
        return 0
    # 平均値を計算する
    mean = np.mean(data)
    print(f"The mean is: {mean}")
    return mean

if __name__ == '__main__':
    my_data = [10, 20, 30, 40, 50]
    calculate_mean(my_data)
\end{lstlisting}

\subsection{アルゴリズムの記述}
`algorithm`と`algpseudocode`パッケージを使うことで、アルゴリズム\ref{alg:euclid}のような擬似コードを記述できます。

\begin{algorithm}[htbp]
  \caption{ユークリッドの互除法 (Euclidean Algorithm)}
  \label{alg:euclid}
  \begin{algorithmic}[1] % [1]で行番号を付ける
    \Require 整数 $a, b$ ($a \ge b > 0$)
    \Ensure $a$と$b$の最大公約数
    \State $r \gets a \pmod b$
    \While{$r \neq 0$}
      \State $a \gets b$
      \State $b \gets r$
      \State $r \gets a \pmod b$
    \EndWhile
    \State \Return $b$
  \end{algorithmic}
\end{algorithm}




\section{結論}
アインシュタインの論文\cite{einstein1905}などを参考に、レポートの結論をここに記述します。 % <-- 変更

% ----------------------------------------------------------------
% ★ 本文が長くなってきたら、body.texを分割して以下のように管理できます
%
% \input{content/01_introduction}
% \input{content/02_methodology}
% \input{content/03_results_and_discussion}
% \input{content/04_conclusion}
%
% このように、main.texを「目次」のように使って章ごとにファイルを分けると、
% さらに管理がしやすくなります。
% ----------------------------------------------------------------


% ----- 付録(Appendix) -----
\appendix
\section{補足データ}

ここには、本文に入れるには長すぎる詳細なデータや、追加の図などを配置します。
このセクションは「付録A」として番号が振られます。
付録内の図や表も、自動的に図目次・表目次に追加されます。

\begin{table}[htbp]
  \centering
  \caption{付録のサンプル表}
  \label{tab:appendix-sample}
  \begin{tabular}{|c|c|}
    \hline
    パラメータ & 値 \\
    \hline
    X & 0.98 \\
    Y & 1.23 \\
    \hline
  \end{tabular}
\end{table}

% -----------------------

% ----- 参考文献リストの挿入 -----
\bibliographystyle{jplain}
\bibliography{bib/references}

\end{document}