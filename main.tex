% ========================================
% LaTeX Template for Japanese Documents
% ========================================
% このファイルは日本語文書作成のためのメインテンプレートです
% 各パッケージの役割と設定について詳しく説明しています

% ========================================
% ドキュメントクラスの設定
% ========================================
\documentclass[dvipdfmx]{article}
% dvipdfmx: 日本語対応のPDF出力エンジン
% article: 一般的な論文・報告書用のドキュメントクラス

% ========================================
% 画像・図表関連のパッケージ
% ========================================
\usepackage{graphicx}
% graphicx: 画像挿入のための基本パッケージ
% \includegraphicsコマンドを使用可能にする

% 画像ファイルの場所をLaTeXに教える
% これにより、\includegraphicsで figures/ フォルダを指定する必要がなくなります
\graphicspath{{figures/}}

% ========================================
% 数式・数学記号関連のパッケージ
% ========================================
\usepackage{amsmath}
% amsmath: 高度な数式環境(align, matrix, cases等)
% 数式の配置や複雑な数式表現を可能にする

\usepackage{amssymb}
% amssymb: 数学記号の拡張(\mathbb, \mathcal等)
% 特殊な数学記号を使用可能にする

\usepackage{bm}
% bm: 太字の数式記号(\bm{数式})
% ベクトルや行列を太字で表現する際に使用

% ========================================
% 図表の配置・キャプション関連
% ========================================
\usepackage{subcaption}
% subcaption: サブ図・サブ表のキャプション
% subfigure環境で複数の図を並べて配置可能

% ========================================
% ソースコード表示関連のパッケージ
% ========================================
\usepackage{listings}
% listings: ソースコードの表示
% シンタックスハイライト、行番号、フレーム等を設定可能

\usepackage{xcolor}
% xcolor: カラー設定
% listingsパッケージで使用する色を定義

% ソースコードの表示設定
\lstset{
  basicstyle=\small\ttfamily,      % 基本スタイル:小さめの等幅フォント
  commentstyle=\color{green},      % コメントの色:緑
  keywordstyle=\color{blue},       % キーワードの色:青
  stringstyle=\color{red},         % 文字列の色:赤
  frame=tb,                        % フレーム:上下に線
  breaklines=true,                 % 長い行を自動改行
  columns=fullflexible,            % 文字幅を調整
  numbers=left,                    % 行番号を左に表示
  numberstyle=\tiny\color{gray},   % 行番号のスタイル:小さく灰色
  showstringspaces=false           % 文字列内のスペースを表示しない
}

% ========================================
% アルゴリズム記述関連のパッケージ
% ========================================
\usepackage{algorithm}
% algorithm: アルゴリズム環境
% アルゴリズムに番号とキャプションを付ける

\usepackage{algpseudocode}
% algpseudocode: 擬似コード記述
% アルゴリズムの擬似コードを記述するための環境

% ========================================
% フロート制御パッケージ(図表の配置制御)
% ========================================
% [section]オプションで、\section の前に自動でバリアを設置
% これにより、図表が次のセクションに流れることを防ぐ
\usepackage[section]{placeins}

% \subsection の前に自動でバリアを設置するためのパッケージ
% より細かいレベルで図表の配置を制御
\usepackage{etoolbox}
\preto{\subsection}{\FloatBarrier}

% ========================================
% ハイパーリンクの設定
% ========================================
\usepackage[
  dvipdfmx,                        % 日本語PDF対応
  colorlinks=true,                 % リンクを色付きで表示
  linkcolor=blue,                  % 内部リンクの色:青
  citecolor=green,                 % 引用リンクの色:緑
  urlcolor=magenta,                % URLの色:マゼンタ
  bookmarks=true,                  % PDFのブックマークを有効化
  bookmarksnumbered=true           % ブックマークに番号を付ける
]{hyperref}

% ========================================
% プロフィール情報の読み込み
% ========================================
% 外部ファイルから著者情報を読み込み
% セキュリティのため、実際のプロフィールファイルはローカルのみに保持
% ========================================
% プロフィール情報設定ファイル
% ========================================

% 基本情報
\newcommand{\profileName}{Koichi Nagasaku}
\newcommand{\profileStudentID}{25HJ046}
\newcommand{\profileLaboratory}{Yamada-Matsui Laboratory}
\newcommand{\profileProgram}{Information Science Program}
\newcommand{\profileSchool}{Graduate School of Environment and Information Sciences}
\newcommand{\profileUniversity}{Yokohama National University}

% 連絡先情報(オプション)
\newcommand{\profileEmail}{nagasaku-koichi-rt@ynu.jp}
\newcommand{\profilePhone}{+81-70-2676-2973}

% 研究分野・専門(オプション)
\newcommand{\profileResearchArea}{機械学習・データサイエンス}
\newcommand{\profileKeywords}{AI, Machine Learning, Data Analysis}

% 報告書タイトル
\newcommand{\reportTitle}{put Your Report Title}

% 報告書の種類(オプション)
\newcommand{\reportType}{課題報告書}
\newcommand{\reportDate}{\today}

% ========================================
% プロフィール情報を表示するコマンド
% ========================================

% 著者情報を生成
\newcommand{\generateAuthor}{
  \profileName \\
  Student ID: \profileStudentID \\
  \profileLaboratory \\
  \profileProgram \\
  \profileSchool \\
  \profileUniversity
}

% 完全なプロフィール情報を表示(オプション)
\newcommand{\showFullProfile}{
  \begin{center}
    \textbf{\large 著者情報}
  \end{center}
  
  \begin{table}[h]
    \centering
    \begin{tabular}{ll}
      \textbf{氏名:} & \profileName \\
      \textbf{学籍番号:} & \profileStudentID \\
      \textbf{研究室:} & \profileLaboratory \\
      \textbf{プログラム:} & \profileProgram \\
      \textbf{研究科:} & \profileSchool \\
      \textbf{大学:} & \profileUniversity \\
      \textbf{研究分野:} & \profileResearchArea \\
      \textbf{キーワード:} & \profileKeywords \\
      \textbf{連絡先:} & \profileEmail \\
    \end{tabular}
  \end{table}
} 

% ========================================
% ドキュメントの基本情報設定
% ========================================
\title{\reportTitle}              % タイトル(profile.texで定義)
\author{\generateAuthor}          % 著者情報(profile.texで定義)
\date{\reportDate}                % 日付(profile.texで定義)

% ========================================
% ドキュメント本文の開始
% ========================================
\begin{document}

% タイトルページの生成
\maketitle

% ========================================
% 要旨(Abstract)セクション
% ========================================
\begin{abstract}
  % 要旨の内容を外部ファイルから読み込み
  ここにレポートの要旨を記述します。研究の背景、目的、手法、結果、結論などを200〜400字程度で簡潔にまとめます。
この部分は独立したページに表示されることが多いです。

\end{abstract}

% ========================================
% 目次・図目次・表目次の生成
% ========================================
\tableofcontents                   % 目次
\listoffigures                     % 図目次
\listoftables                      % 表目次
\newpage                           % 改ページ

% ========================================
% 本文セクション
% ========================================
% 本文の内容を外部ファイルから読み込み
% ----- プロフィール情報の表示(オプション) -----
% 以下の行のコメントアウトを外すと、詳細なプロフィール情報が表示されます
% \showFullProfile
% \newpage
% ------------------------------------------------

\section{はじめに}

本文の執筆は、このファイルで行います。
LaTeXは高機能な文書作成システムであり\cite{lamport1994}、特に数式や参考文献の扱いに優れています。
詳細は公式サイト \href{https://www.latex-project.org/}{The LaTeX Project} や、簡単な紹介サイト \url{https://texwiki.texjp.org/} などで確認できます。




\section{基本的な要素の例}

\subsection{図の挿入}
`figure`環境を用いて、図\ref{fig:sample}のような画像を挿入できます。
`main.tex`に`\graphicspath{{figures/}}`と記述したため、画像ファイルは`figures`フォルダに置いてください。

\begin{figure}[htbp]
  \centering
  \includegraphics[width=0.7\textwidth]{example-image-a}
  \caption{基本的な図の例}
  \label{fig:sample}
\end{figure}

\subsection{表の作成}
表\ref{tab:sample}に、`tabular`環境を用いて作成した表の例を示します。

\begin{table}[htbp]
  \centering
  \caption{サンプル表}
  \label{tab:sample}
  \begin{tabular}{|l|c|r|}
    \hline
    \textbf{左揃え (l)} & \textbf{中央揃え (c)} & \textbf{右揃え (r)} \\
    \hline
    商品A & 150個 & 30,000円 \\
    商品B & 20個 & 5,400円 \\
    商品C & 350個 & 105,000円 \\
    \hline
  \end{tabular}
\end{table}

\subsection{グラフの挿入}
グラフも図と同様に、外部ツールで作成した画像ファイルを挿入するのが一般的です(図\ref{fig:graph-image})。

\begin{figure}[htbp]
  \centering
  \includegraphics[width=0.8\textwidth]{example-image-b}
  \caption{外部ツールで作成し、画像として挿入したグラフの例}
  \label{fig:graph-image}
\end{figure}




\section{技術的な要素の記述例}

\subsection{Subfigureによる図の並列配置}
`subcaption`パッケージを使うと、図\ref{fig:subfigures}のように複数の図を並べて配置し、それぞれに(a), (b)のような副キャプションを付けることができます。

\begin{figure}[htbp]
  \centering
  % 1つ目の図 (a)
  \begin{subfigure}[b]{0.48\textwidth}
    \centering
    \includegraphics[width=\textwidth]{example-image-a}
    \caption{1つ目の図(結果A)}
    \label{fig:sub-a}
  \end{subfigure}
  \hfill % 図の間に少しスペースを空ける
  % 2つ目の図 (b)
  \begin{subfigure}[b]{0.48\textwidth}
    \centering
    \includegraphics[width=\textwidth]{example-image-b}
    \caption{2つ目の図(結果B)}
    \label{fig:sub-b}
  \end{subfigure}
  \caption{2つの結果を比較した図}
  \label{fig:subfigures}
\end{figure}

\subsection{ソースコードの掲載}
`listings`パッケージを用いると、ソースコード\ref{lst:pythonsample}のように、シンタックスハイライトや行番号を付けてコードを綺麗に表示できます。

\begin{lstlisting}[language=Python, caption={Pythonによる簡単なサンプルコード}, label={lst:pythonsample}]
# This is a sample Python code.
import numpy as np

def calculate_mean(data):
    """Calculate the mean of a list of numbers."""
    if not data:
        return 0
    # 平均値を計算する
    mean = np.mean(data)
    print(f"The mean is: {mean}")
    return mean

if __name__ == '__main__':
    my_data = [10, 20, 30, 40, 50]
    calculate_mean(my_data)
\end{lstlisting}

\subsection{アルゴリズムの記述}
`algorithm`と`algpseudocode`パッケージを使うことで、アルゴリズム\ref{alg:euclid}のような擬似コードを記述できます。

\begin{algorithm}[htbp]
  \caption{ユークリッドの互除法 (Euclidean Algorithm)}
  \label{alg:euclid}
  \begin{algorithmic}[1] % [1]で行番号を付ける
    \Require 整数 $a, b$ ($a \ge b > 0$)
    \Ensure $a$と$b$の最大公約数
    \State $r \gets a \pmod b$
    \While{$r \neq 0$}
      \State $a \gets b$
      \State $b \gets r$
      \State $r \gets a \pmod b$
    \EndWhile
    \State \Return $b$
  \end{algorithmic}
\end{algorithm}




\section{結論}
アインシュタインの論文\cite{einstein1905}などを参考に、レポートの結論をここに記述します。

% ========================================
% ファイル分割の例(コメントアウト)
% ========================================
% ★ 本文が長くなってきたら、body.texを分割して以下のように管理できます
%
% \input{content/01_introduction}        % 1章:はじめに
% \input{content/02_methodology}         % 2章:手法
% \input{content/03_results_and_discussion} % 3章:結果と考察
% \input{content/04_conclusion}          % 4章:結論
%
% このように、main.texを「目次」のように使って章ごとにファイルを分けると、
% さらに管理がしやすくなります。
% 各章のファイルは content/ フォルダに配置することを推奨します。

% ========================================
% 付録(Appendix)セクション
% ========================================
\appendix                          % 付録環境の開始
\section{補足データ}

ここには、本文に入れるには長すぎる詳細なデータや、追加の図などを配置します。
このセクションは「付録A」として番号が振られます。
付録内の図や表も、自動的に図目次・表目次に追加されます。

\begin{table}[htbp]
  \centering
  \caption{付録のサンプル表}
  \label{tab:appendix-sample}
  \begin{tabular}{|c|c|}
    \hline
    パラメータ & 値 \\
    \hline
    X & 0.98 \\
    Y & 1.23 \\
    \hline
  \end{tabular}
\end{table}
           % 付録の内容を外部ファイルから読み込み

% ========================================
% 参考文献リストの挿入
% ========================================
\bibliographystyle{jplain}         % 参考文献のスタイル(日本語対応)
\bibliography{bib/references}      % 参考文献ファイルの指定

% ========================================
% ドキュメントの終了
% ========================================
\end{document}